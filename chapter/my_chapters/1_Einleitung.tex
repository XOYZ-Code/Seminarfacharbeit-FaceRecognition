% „ = Alt + 0132
% “ = Alt + 0147
\section{Einleitung}
\label{sec:einleitung}
Vor ungefähr 64 Jahren wurde das erste Programm, welches speziell zur Problemlösung und Nachahmung eines Menschen entwickelt wurde, von Herbert Alexander Simon und Allen Newell geschrieben. Dieses Programm war Quellen zufolge das erste Programm welches der Definition von Künstlicher Intelligenz, ein Programm welches speziell dafür entwickelt wurde die Fähigkeiten eines Menschen Nachzuahmen, gerecht wurde [vgl. \ref{bib:LogicTheorist}] wobei zu dem Zeitpunkt der Entwicklung dieser das Wort "Künstliche Intelligenz" [folgend auch als KI bezeichnet] noch nicht existierte bzw. nicht in diesem Kontext verwendet wurde [vgl. \ref{LogicTheorist:Wikipedia}]. Seitdem hat sich das Thema KI sowohl in der Hardware aber auch in der Software stark weiterentwickelt. Wo Künstliche Intelligenzen früher nur mit einfachen Zahlen arbeiten konnten, können einige mittlerweile Personen, Gesichter, Objekte und andere Dinge identifizieren und differenzieren. Doch auch hierbei gibt es Probleme die seit der Entwicklung von Gesichtserkennungsprogrammen in den 1960er nicht vollständig gelöst werden konnten, weshalb es bisher auch keine KI gibt welche keine Mängel aufweist. Und gibt es eine KI die momentan eine Erfolgschance von 100\% hat ist dies meist aus dem Grund so, dass sie ihrem Schwachpunkt noch nicht begegnet ist. In dieser Facharbeit wird das Thema der Gesichtserkennung, zu engl. "Face recognition", behandelt. Bei dieser Art der Anwendung von Künstlicher Intelligenz liegt die Schwierigkeit besonders bei Unschärfe, Fragmenten, Verzerrungen und Ähnlichkeiten von Objekten und besonders von Gesichtern. Dies ist z.B. bei Zwillingen der Fall an dem viele Systeme noch heute scheitern. Zudem wird die Theorie von Face Recognition mit Hilfe von Künstlicher Intelligenz erklärt, mit praktischen Beispielen erläutert und auf ethischer Basis analysiert.
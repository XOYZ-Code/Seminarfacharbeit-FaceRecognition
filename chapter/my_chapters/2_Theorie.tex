% Theorie Einführung
\section{Theoretischer Hintergrund}
\label{sec:theorie}
    \subsection{Definition Künstliche Intelligenz}
    \label{subsec:definiton_kuenstliche_intelligenz}
    Der Interesse an dem Thema Künstliche Intelligenz sowie die Einsatzmöglichkeiten steigen stetig an. Künstliche Intelligenz ist ein sehr großes Thema. Für eine grobe Wiedergabe worum es in diesem Thema geht liefert Wikipedia einen guten Einstieg: 
    \textit{\enquote{Künstliche Intelligenz [...] ist ein Teilgebiet der Informatik, welches sich mit der Automatisierung intelligenten Verhaltens und dem maschinellen Lernen befasst.}} [siehe \ref{wiki:KuenstlicheIntelligenz}]\\

    Mit dieser Aussage kann man Künstliche Intelligenz im groben Beschreiben. Etwas genauer beschreibt es Klaus Mainzer in seinem Buch \enquote{Künstliche Intelligenz - Wann übernehme die Maschinen?} [siehe \ref{book:KI_WannUebernehmenDieMaschinen}]. In diesem geht er auf die Definition der Intelligenz von Systemen ein. Laut ihm heißt ein System intelligent \textit{\enquote{wenn es selbstständig und effizient Probleme lösen kann. Der Grad der Intelligenz hängt vom Grad der Selbstständigkeit, dem Grad der Komplexität des Problems und dem Grad der Effizienz des Problemlösungsverfahrens ab}} [siehe S.3 in \ref{book:KI_WannUebernehmenDieMaschinen}].\\
    
    Um es einmal auf den Punkt zu bringen, bezeichnet \enquote{Künstliche Intelligenz} ein System welches selbstständig lernen und effizient Probleme lösen kann und zudem noch lernfähig ist.

    \subsection{Wie funktioniert eine Künstliche Intelligenz?}
    \label{subsec:wie_funktioniert_eine_kuenstliche_intelligenz}
    Künstliche Intelligenzen funktionieren je nach Abwandlung etwas verschieden aber ähneln sich dennoch in ihrem Aufbau. In dieser Arbeit werden die sogenannten Neuronalen Netzwerke, welche in den meisten Systemen Anwendung finden, behandelt. Dennoch werden zunächst die beiden Definitionen \enquote{Deep Learning} und \enquote{Machine Learning} behandelt, da diese eine große Rolle im Bereich KI spielen.

        \subsubsection{Deep Learning}
        \label{subsubsec:deep_learning}
            Deep Learning ist ein Lernverfahren für eine KI welches selbst auch als Machine Learning bezeichnet werden kann. In einem solchen Verfahren lernt die Software mit Hilfe eines programmierten Neuronalen Netzwerk um im Nachhinein präzise Ausgaben zu tätigen [vgl. \ref{subsec:machine_learning_and_deep_learning}]. Diese spezielle Art des Lernens wird für die Neuronalen Netze verwendet die in dieser Arbeit hauptsächlich behandelt werden. In einem späteren Abschnitt wird das Trainieren eines Neuronalen Netzes noch einmal genauer erklärt. [siehe \ref{subsubsec:trainieren_eines_neuronalen_netzes}]

        \subsubsection{Machine Learning}
        \label{subsubsec:machnine_learning}
            \enquote{Algorithms that parse data, learn from that data, and then apply what they’ve learned to make informed decisions} [siehe \ref{subsec:machine_learning_and_deep_learning}]. Sinngemäß übersetzt heißt dieser Satz \enquote{Algorythmen die Daten analysieren, von diesen lernen, und ihr erlerntes Wissen zu nutzen um sachkundige Entscheidungen zu treffen}. Dies ist die Grunddefinition zu Machine Learning, zu deutsch Maschinelles Lernen, auf die sich in dieser Arbeit berufen wird.\\
            Um dies an einem realen Beispiel zu veranschaulichen kann man sich Streaming Dienste wie Spotify, Deezer, Netflix oder Amazon Prime nahmen. Hierbei sei gesagt, dass die folgende Theorie zur Verarbeitung von Daten nicht unbedingt der Wirklichkeit entsprechen muss. Dies ist nur ein Ansatz auf welche Art und Weise ein Machine Learning Ansatz umsetzbar ist.\\

            Für das Maschinelle Lernen ist keine komplexe Struktur nötig. Wenn man einfach nur ein System möchte welches einem Nutzer z.B. auf der Seite Netflix neue Serien und Filme, basierend auf den vergangen geschauten Inhalten, vorschlägt und man keinen Wert auf einen zu 100\% Leistungsoptimierten Ansatz vorraussetzt, so kann man mit einfachen Tags und Textdateien arbeiten. Tags ist Englisch und bedeutet Etikett oder Stichwort. Einfach gesagt werden diese Tags an alles angeheftet was man analysieren und kategorisieren möchte. Man kann diese mit den sogenannten Hashtags von Instagram, Twitter und anderen sozialen Netzwerken vergleichen. Diese werden auch zum kategorisieren verwendet. Sucht ein Nutzer z.B. nach Katzen oder dem Hashtag Katzen werden ihm Inhalte bevorzugt angezeigt die genau diesen tag, also mit dem Etikett \enquote{Keatze} versehen sind, versehen sind. Nehmen wir für unser Beispiel einen Nutzer, welcher folgende Serien in seinem Verlauf hat. In der linken Spalte der vorliegenden Tabelle wird dabei der Serien Name angezeigt und in der rechten Spalte werden die Tags zu der Serie angezeigt (diese können von den tatsächlichen Tags abweichen und dienen nur zur veranschaulichung)
            
            \begin{center}
                \begin{tabular}[h]{l|l}
                \label{Tabelle_1}
                    Serie & Tags \\
                    \hline
                    The Walking Dead & Zombies, Apokalypse, Überleben, Action \\
                    Arrow & DC-Comics, Action, Stephen Amell \\
                    World Trigger & Anime, Action, Dimensionen \\
                \end{tabular}
            \end{center}

            Dies sind zwar nur wenig Daten aber zum Verständnis reicht es vollkommen aus. Nehmen wir nun die einzelnen Tags, können wir sehen bzw. vermuten welche Serien der Nutzer gerne schaut und ihm dementsprechend Empfehlungen geben. Hier in einem Säulendiagramm dargestellt.

            \begin{center}
                \begin{tikzpicture}
                \label{Diagramm_1}
    
                    \draw (0cm,0cm) -- (11cm,0cm);
                    \draw (0cm,0cm) -- (0cm,-0.1cm);
                    \draw (11cm,0cm) -- (11cm,-0.1cm);
                    
                    \draw (-0.1cm,0cm) -- (-0.1cm,3cm);
                    \draw (-0.1cm,0cm) -- (-0.2cm,0cm);

                    \foreach \x in {1,...,3}
                        \draw[gray!50, text=black] (-0.2 cm,\x cm) -- (11 cm,\x cm) node at (-0.5 cm,\x cm) {\x};

                    \foreach \x/\y/\tag in {
                        0.5/3/Action,
                        2/1/Anime,
                        3.5/1/Apokalypse,
                        5/1/DC-Comics,
                        6.5/1/Dimensionen,
                        8/1/Stephen Amell,
                        9.5/1/Überleben,
                        11/1/Zombies
                    } {
                        \draw[fill=myblue] (\x cm,0cm) rectangle (1cm+\x cm,\y cm) 
                        node at (0.5cm + \x cm,\y cm + 0.3cm) {\y};
                        \node[rotate=45, left] at (0.6 cm +\x cm,-0.1cm) {\tag};
                    }

                
                \end{tikzpicture}
            \end{center}

            In einer für und Menschen anschaulichen Version dargestellt, ist nun deutlich zu sehen, dass der Nutzer Action-Serien priorisiert, denn im Gegensatz zu aderen Tags ist der Tag Action drei mal vorgekommen. Wie gesagt ist dies eine sehr ungenaue Vorhersage aufgrund der geringen Datenmenge. Nutzt man den jeweiligen Service aber nun schon eine Weile kann man daraus sehr interessant Daten gewinnen. Aber was bringt uns nun zu wissen welche Tags häufig im Serienverlauf des Nutzers vorkommen? Ganz einfach. Wenn in einem Monat z.B. 40 neue Serien auf Netflix lizenziert werden, ist es sinnlos dem Nutzer einfach alle auf dem Startbildschirm zu zeigen. Sinnvoller wäre es dem Nutzer erst die Serien zu zeigen die mit ähnlichen Tags markiert sind wie diese die der Nutzer mag. So kann in der Theorie für den Nutzer ein spannender und nicht langweiliger Filmabend gesichert werden. Zudem werden diese Informationen an Marketing Firmen weiterverkauft, wodurch versucht wird den Nutzer zu neuen Inhalten zu bewegen die seinem Interessengebiet entsprechen [vgl. \ref{Netflix_Research:machine_learning_learn_how_to_entertain_the_world}]

        \subsubsection{Aufbau eines Neuronalen Netzes}

            Neuronale Netze sind nach der Theorie wie unser Gehirn funktioniert aufgebaut. Hierbei arbeiten viele Neurone zusammen um Daten und Befehle zu verarbeiten [vgl. \ref{subsubsec:dasgehirn:Zellen-Arbeiter_Des_Gehrins}]. Die Neuronalen Netze besitzen 3 verschiedene Layer welche zur Datenverarbeitung genutzt werden können.\\

            \begin{figure}[ht]
                \includegraphics[width=\textwidth, scale=0.8]{resources/images/img/Neural Networks/neural-network-example.png}
                \caption{Beispiel eines Neuronalen Netzes [Quelle: \ref{subsubsec:Beispiel_eines_Neuronalen_Netzes}]}
                \label{fig:Beispiel_Neuronales_Netz}
            \end{figure}
            
            Diese Layer werden \enquote{Input Layer}, \enquote{Hidden Layer} und \enquote{Output Layer} genannt. Innerhalb dieser gibt es sogenannte Neuronen, welche mit jeweils jedem Neuron der vorherigen und der nächsten Schicht verbunden ist. Jedes dieser Neuronen hat ein Gewicht welches das Ergebnis im \enquote{Output Layer} beeinflussen kann. Wie viele Neuron-Verbindungen es gibt hängt von der Anzahl der Neuronen sowie der \enquote{Hidden Layers} ab. Im Beispiel [Abbildung \ref{fig:Beispiel_Neuronales_Netz}] ist diese Rechnung noch sehr simpel.
            
            \begin{align}
                3 \cdot 4 = 12 + (4 \cdot 4) = 28 + (4 \cdot 1) = 32
            \end{align}
            
            Das Beispiel hat also insgesamt 32 Neuronen Verbindungen. Dies ist noch ein sehr kleines Netz, welches für Anschauungszwecke allerdings völlig ausreicht.\\
        
        \subsubsection{Trainieren eines Neuronalen Netzes}
        \label{subsubsec:trainieren_eines_neuronalen_netzes}
            Mit einem gerade erstellten Neuronalen Netz können noch keine Daten sinnvoll verarbeitet werden. Zunächst muss dieses auf das Anwendungsgebiet trainiert werden.\\
            In diesem Training bekommen die Neuronen zunächst ein zufälliges Gewicht welches meist zwischen -1 und 1 liegt. Als nächstes werden Daten in das Neuronale Netz gegeben. Natürlich wird die Ausgabe größtenteils falsch sein und nur durch Glück einen richtigen Ansatz haben. Nach dem der erste Datensatz durchgelaufen ist, passt das Netz seine Gewichte automatisch an. Hierbei erkennt es welche Neuronen den größten Einfluss auf das aktuelle Ergebnis hatten und berechnet dabei die Abweichung vom erwarteten Ergebnis die diese verursachten. Im Anschluss werden diese fehlerhaften Gewichte ein kleines bisschen angepasst sodass die Ergebnisse am \enquote{Output Layer} ein wenig näher am erwarteten Ergebnis sind als vorher [vgl. \ref{subsubsec:Wie_lernen_neuronale_netze}]. Dieser Vorgang wird nun mit einem großen Datensatz mehrere hunderte oder tausend mal wiederholt. Hierbei gilt je größer der Datensatz desto genauer und \enquote{intelligenter [vgl. \ref{subsec:definiton_kuenstliche_intelligenz}]} das Neuronale Netz.
    
    \subsection{Erste Konzepte Künstlicher Intelligenz}
    \label{subsec:Theorie:Erste_Konzepte_von_KI}
        Wie in der Einleitung schon erwähnt, begann die Entwicklung von Künstlicher Intelligenz schon vor mehreren Jahrzehnten. Am Anfang gab es die Bezeichung \enquote{Künstliche Intelligenz} noch nicht. Dennoch schafften es die beiden Wissenschaftler Herbert Alexander Simon und Allen Newell ein Lernfähiges Computersystem zu erschaffen, welches einen menschlich-orientierten Lösungsansatz verwendete [Bezug zu \ref{sec:einleitung}]. Dieses System hieß \enquote{Logic Theorist}
        
        \subsubsection{Logic Theorist - Was konnte es?}
        \label{subsubsec:Logic_Theorist:Was_konnte_es}
            Der Logic Theorist sollte dazu dienen Mathematische Behauptungen zu beweisen. Von dem zweiten Kapitel der \enquote{Prinzipien der Mathematik}, ein Buch Trilogie welche Grundlagen der Mathematik zusammenfasst [siehe \ref{book:prinzipa_of_mathmatics}], konnte das Programm von den ersten 52 Theorien insgesamt 38 beweisen. Einige dieser Lösungen waren laut Quellen sogar \enquote{schöner} gelöst als die handschriftliche Lösung der Autoren Bertrand Russel und Alfred North Whiteheard.

    \subsection{Das System der Gesichtsentdeckung}
    \label{subsec:system_of_face_detection}
        Die Gesichtsentdeckung, zu englisch \enquote{face detection}, bildet die Basis der Erkennung bzw. Wiedererkennung von Gesichtern. Um ein Gesicht in einem Bild oder Video erkennen zu können, müssen zunächst die biometrischen Daten der Gesichter auf dem Bild extrahiert werden. Dabei werden Bilder systematisch nach Gesichtern untersucht sowie die Lage, die Pixelkoordinate im Bild, der dabei gefundenen Merkmale gespeichert. Hierbei gibt es verschiedene Methoden die in den folgenden Unterkapiteln behandelt werden [vgl. \ref{book:DigitalGesichtserkennung}]. Hierbei sei zu beachten, dass es einen Unterschied zwischen der Gesichtsentdeckung und der Gesichtserkennung gibt. Während die Gesichtsentdeckung nur dazu dienen soll Gesichter zu erkennen, geht die Gesichtserkennung noch weiter und kann diese analysieren und zuordnen.

        \subsubsection{Gesichtsentdeckung durch Bildreduzierung um Hintergrundinformationen}
        \label{subsubsec:face_detection_bildreduzierung_um_hintergrundinformationen}
            Wenn in einem Bild der Hintergrund bekannt ist, z.B. von einer Kamera die auf einem Stativ still an einem Ort verweilt, können Gesichter durch Substraktion des gespeicherten Hintergrundes und des aufgenommenen Bildes aus dem Bild extrahiert werden. Hierbei ist wichtig, dass das Gesicht nur in der Frontalansicht ordnungsgemäß erkannt werden kann. Zudem ist es wichtig, dass sich der Hintergrund nicht verändert und durchgehend statisch bleibt. [vgl. \ref{book:DigitalGesichtserkennung}]
        
        \subsubsection{Gesichtsentdeckung durch Farbinformationen}
        \label{subsubsec:face_detection_with_color_information}
            Das Menschliche Gesicht kann in einem Bild auf einer relativ großen Fläche in einem Bild eine sehr monotone Farbgebung darstellen. Dies kann man sich zu nutze machen, und in einem Bild nach diesen Ansammlungen suchen. Dazu wird zunächst eine Analyse gestartet, welche die Sättigung pro Farbton angibt. Durch die relativ dichte und monotone Farbgebung eines Gesichtes, kann durch dieses Analyseverfahren auf ein Gesicht geschlossen werden.\\
            Diese Methode ist allerdings nicht perfekt und birgt zwei große nachteile. Zunächst gibt es das Problem mit den Lichtverhältnissen. In dunklen Bildern werden Menschen mit dunkler Hautfarbe eher weniger erkannt als Menschen mit heller Hautfarben sowie in hellen Bildern eher dunkelhäutige Menschen erkannt werden. Außerdem gibt es das Problem, dass große Ansammlungen von ähnlichfarbigen Pixeln die Ausgabe beeinflussen können. [\ref{book:DigitalGesichtserkennung}]

        \subsubsection{Gesichtsentdeckung durch Bewegungsinformationen}
        \label{subsubsec:face_detection_durch_bewegungsinformationen}
            So wie Gesichter durch Farbinformationen erkannt werden können sie ähnlich auch durch Bewegungsinformationen erkannt werden. Diese art der Gesichtsentdeckung ist aufgrund der benötigten Daten auf Videos beschränkt. In diesem Verfahren wird sich zu nutze gemacht, dass das Gesicht fast dauerhaft in bewegung ist. Der Algorithmus vergleicht die korrespondierenden Pixel und geht von einem neuen Bild aus, wenn diese sich um einen bestimmten Betrag verändern. Aus diesen wird dann eine Differenz gebildet wodurch ein Gesicht erkannt werden kann. Welcher Bildausschnitt betrachtet werden soll, wird im vorhinein festgelegt. Allerdings produziert dieses Verfahren ene vielzahl an Fehlern, da auch bewegte Objekte als \enquote{Gesichter} erkannt werden [\ref{book:DigitalGesichtserkennung}].

        \subsubsection{Gesichtsentdeckung durch Geometrie}
        \label{subsubsec:face_detectiom_with_geometry}
            Face detection durch Geometrie ist wohl mit eine der genausten Arten Gesichtsentdeckung durchzuführen. Hierbei wird ein Bild soweit gefiltert, dass es nur noch eine Binäre Kantenebene hat. Das bedeutet, dass z.B. von einem Gesicht nur noch die Umrisse von Augen, Nase, Mund, Haare und Form sichtbar sind. Dieses Kantenbild, wird dann mit dem Kantenbild eines Typischen Gesichts verglichen indem die Hausdorff-Distanz berechnet wird. Diese Distanz beschreibt den kleinstmöglichen Abstand einer Punktmenge zu einer zweiten Punktmenge. Dieses Verfahren führt man dann mit einem detaillierten Bild einer Augenpartie durch und kann dann, aufgrund der Hausdorff-Differenz, entscheiden ob es sich um ein Gesicht handelt oder nicht [vgl. \ref{book:DigitalGesichtserkennung}].
            
    \subsection{Das System der Gesichtserkennung}
    \label{subsec:system_of_face_recognition}
        Noch weiter als die Gesichtsentdeckung geht die Gesichtserkennung. Sie beschäftigt sich nicht nur mit der Erkennung von Gesichtern, sondern auch mit der Identifizierung von Personen anhand biometrischer Daten [vgl. \ref{book:DigitalGesichtserkennung} \& \ref{source:searchenterpriseai:facial_recognition}]. Hierbei wird zwischen der Lernphase und Erkennungsphase unterscheidet. In diesem Abschnitt wird auf die Funktionsweise verschiedener Algorithmen eingegangen, die zur Gesichtserkennung genutzt werden.\\
        In den Anfängen der Gesichtserkennung wurde ein Bild nach Augen, Mund und Nase durchsucht. Von diesen wurden die relativen Abstände bestimmt wodurch ermittelbar war ob es sich um ein Gesicht handelt oder nicht. Der Nachteil dieser Methode (welche auch 2D-Verfahren genannt wird) ist, dass nur Frontalaufnahmen von Gesichtern ordnungsgemäß erkannt werden konnten.

        \subsubsection{3D-Verfahren}
        \label{subsubsec:3d_verfahren}
            Im Gegensatz zum 2D-Verfahren nutzt das 3D-Verfahren den 3-dimensionalen Raum. Hierbei wird nicht einfach nur der Abstand von Augen, Nase, Mund und anderen Gesichtsmerkmalen auf der X und Y-Achse bemessen sondern mit Hilfe von Streifenprojektion auch auf der Z-Achse bemessen. Durch diese drei Achsen ist es auch möglich Winkelberechnungen anzustellen welche bei der Differenzierung von verschiedenen Gesichtern nützlich sein kann. Für diese Methode sind allerdings auch spezielle Sensoren notwendig [vgl. \ref{book:DigitalGesichtserkennung} \& \ref{wiki:face_recognition}]. Diese sind mittlerweile aber auch schon in vielen Smartphones verbaut. Eine genauere Erklärung dieser Sensoren ist in dem Abschnitt [\ref{subsec:FaceID_TrustedFace}] zu finden.

        \subsubsection{Elastic bunch Graph Matching}
        \label{subsubsec:Elastic_bunch_graph_Matching}
            

            
\section{Literatur Verzeichnis}
\label{sec:literatur}

\subsection{Logic Theorist}
\label{bib:LogicTheorist}
	Letzter Zugriff: 12.02.2020\\
	URL: https://history-computer.com/ModernComputer/Software/LogicTheorist.html

	\subsubsection{Memorandum}
	\label{LogicTheorist:Memorandum}
		Letzter Zugriff: 18.02.2020 12:38\\
		URL: https://history-computer.com/Library/Logic\%20Theorist\%20memorandum.pdf
		
	\subsubsection{Wikipedia Artikel}
	\label{LogicTheorist:Wikipedia}
	Letzter Zugriff: 18.02.2020\\
	URL: https://en.wikipedia.org/wiki/Logic\_Theorist

\subsection{Allgemeine Quellen zum Thema KI}
\label{bib:AllgeimeineQuellen}
		\subsubsection{Wikipedia Artikel zum Thema Künstliche Intelligenz}
		\label{wiki:KuenstlicheIntelligenz}
		Letzter Zugriff: 19.02.2020 12:25\\
		URL: https://de.wikipedia.org/wiki/Künstliche\_Intelligenz

		\subsubsection{Gesichtserkennung}
		\label{wiki:face_recognition}
		Letzter Zugriff: 02.03.2020 17:19\\
		URL: https://de.wikipedia.org/wiki/Gesichtserkennung
	
\subsection{Aufbau und Funktionsweise Neuronaler Netze}		
\label{subsec:Aufbau_Funktion_Neuronaler_Netze}
		\subsubsection{JAAI - Aufbau von Neuronalen Netzen}
		\label{subsubsec:Aufbau_von_Neuronalen_Netzen}
		Letzter Zugriff: 22.02.22 15:04\\
		URL: https://jaai.de/kuenstliche-neuronale-netze-aufbau-funktion-291/

		\subsubsection{Wie lernen Neuronale Netze?}
		\label{subsubsec:Wie_lernen_neuronale_netze}
		Letzter Zugriff: 23.02.2020 17:49\\
		URL: https://jaai.de/machine-deep-learning-529/

\subsection{Bücher}
\label{books}

	\subsubsection{Buch: Digitale Gesichtserkennung}
	\label{book:DigitalGesichtserkennung}
		Untertitel: Theoretischer Überblick und praktische C++-Implementierung\\
		Autor: Andreas G. Ranftl\\
		Jahr: 2012\\
		Ort: Hamburg\\
		ISBN: 978-3-86341-432-0\\

	\subsubsection{Buch: Künstliche Intelligenz - Wann übernehmen die Maschinen?}
	\label{book:KI_WannUebernehmenDieMaschinen}
		Autor: Klaus Mainzer\\
		Jahr: 2016\\
		ISBN: 978-3-662-48452-4\\
		ISBN: 978-3-662-48453-1 (eBook)

	\subsubsection{Buch: Prinzipien der Mathematik}
	\label{book:prinzipa_of_mathmatics}
		Letzter Zugriff: 24.02.2020 12:14\\
		Autor: Bertrand Russel, Alfred North Whitehead\\
		URL: https://plato.stanford.edu/entries/principia-mathematica/

	\subsection{Buch: Religionsbuch Oberstufe}
	\label{book:religionsbuch_oberstufe}
		Herausgeber: Prof. Dr. Ulrike Baumann und Prof. Dr. Friedrich Schweitzer\\
		Ort: 2014 Berlin\\
		Verlag: Cornelsen Verlag\\

\subsection{Informationen über das menschliche Gehirn}
\label{subsec:Informationen_ueber_das_menschliche_Gehirn}
	\subsubsection{dasgehirn.info - Zellen: spezialisierte Arbeiter des Gehirns}
	\label{subsubsec:dasgehirn:Zellen-Arbeiter_Des_Gehrins}
	Letzter Zugriff: 22.02.2020 18:03\\
	URL: https://www.dasgehirn.info/grundlagen/kommunikation-der-zellen/zellen-spezialisierte-arbeiter-des-gehirns

\subsection{Viola Jones Algorithmus}
\label{algoryhtm:viola_jones}
	Letzter Zugriff: 02.03.2020 12:52\\
	URL: https://pdfs.semanticscholar.org/40b1/0e330a5511a6a45f42c8b86da222504c717f.pdf

\subsection{Facial Recognition}
\label{source:searchenterpriseai:facial_recognition}
	Letzter Zugriff: 02.03.2020 13:47\\
	URL: https://searchenterpriseai.techtarget.com/definition/facial-recognition

	\subsubsection{Elastic Bunch Graph Matching}
	\label{scholarpedia:ebgm}
	Letzter Zugriff: 02.03.2020 19:42\\
	URL: http://www.scholarpedia.org/article/Elastic\_Bunch\_Graph_Matching

	\subsubsection{Gabor_wavelets}
	\label{Gabor_wavelets}
	Letzter Zugriff: 02.03.2020 20:53\\
	URL: https://en.wikipedia.org/wiki/Gabor\_wavelet

\subsection{\enquote{Machine Learning} und \enquote{Deep Learning}}
\label{subsec:machine_learning_and_deep_learning}
	Letzter Zugriff: 01.03.2020 22:22\\
	URL: https://www.zendesk.com/blog/machine-learning-and-deep-learning/

\subsection{Netflix Research}
\label{Netlfix_Research}
	\subsubsection{Machine Learning - Learn hoe to entertain the World}
	\label{Netflix_Research:machine_learning_learn_how_to_entertain_the_world}
	Letzter Zugriff: 02.03.2020\\
	URL: https://research.netflix.com/research-area/machine-learning

\subsection{Dynamic Link Machine}
\label{subsec:dynamic_link_machine}
	Letzter Zugriff: 02.03.2020 22:55\\
	URL: http://www.cs.utexas.edu/users/nn/web-pubs/htmlbook96/

\subsection{Apples Support - Fortschritt von Face ID}
\label{apple:face_id}
	Letzter Zugriff: 03.03.2020 03:28\\
	URL: https://support.apple.com/de-de/HT208108

\subsection{Face ID - Face ID zu langsam?}
\label{maclife:face_id_too_slow}
	Letzter Zugriff: 03.03.2020 04:00\\
	URL: https://www.maclife.de/news/entsperrt-face-id-iphone-schnell-genug-oder-langsam-100111801.html

	\subsubsection{Vergleich von Touch ID und Face ID}
	\label{cickrepair:touch_id_vs_face_id}
		Letzter Zugriff: 03.03.2020 04:08\\
		URL: https://www.clickrepair.de/ratgeber/ratgeber-iphone/ratgeber-iphone-tipps/vergleich-von-face-id-und-touch-id-was-ist-besser

\subsection{Massenüberwachungsstaat China}
	\subsubsection{Welt.de: Ein Sputnik-Moment macht China zur KI-Weltmacht}
	\label{welt:China_ki_weltmacht}
		Letzter Zugriff: 03.03.2020 04:19\\
		URL: https://www.welt.de/kultur/article191734655/Wie-China-mit-kuenstlicher-Intelligenz-zum-Ueberwachungsstaat-wird.html

	\subsubsection{Berliner Zeitung - Totale Überwachung in China}
	\label{berlinerZeitung:totale_ueberwachung_in_china}
		Letzter Zugriff: 03.03.2020 04:21\\
		URL: https://www.berliner-zeitung.de/zukunft-technologie/kuenstliche-intelligenz-totale-ueberwachung-ist-in-china-laengst-normalitaet-li.37733

	\subsubsection{Handelsblatt: Chinesen schätzen die Überwachung}
	\label{handelsblatt:chinesen_schaetzen_die_ueberwachung}
		Letzter Zugriff: 03.03.2020 04:44\\
		URL: https://www.handelsblatt.com/politik/international/totale-ueberwachung-darum-befuerworten-viele-chinesen-das-sozialpunktesystem/22834722.html?ticket=ST-1728959-Kf2JHUFOchRnqngAmVYS-ap3

\subsection{Face Recognition mit Python in under 25 Lines of code}
\label{realpython:face_recognition_in_under_25_lines_of_code}
	Letzter Zugriff: 03.03.2020\\
	URL: https://realpython.com/face-recognition-with-python/

\subsection{Bilder}
\label{subsec:Bilder_Anhang}
	\subsubsection{Beispiel eines Neuronalen Netzes}
	\label{subsubsec:Beispiel_eines_Neuronalen_Netzes}
	Download am: 22.02.2020 15:20\\
	URL: https://de.cleanpng.com/png-jxw2np/

	\subsubsection{Datenpunkte eines EGBM-Algorithmus}
	\label{image:datapoints_of_a_EGBM_algoryhtm}
	Screenshot am: 02.03.2020 19:17\\
	Von URL: http://www.scholarpedia.org/article/Elastic\_Bunch\_Graph\_Matching

	\subsubsection{Bunch of Jets}
	\label{image:Bunch_of_jets}
		Download am: 03.03.2020 00:56\\
		URL: http://www.scholarpedia.org/article/File\:BunchGraph.png

	\subsection{Gewissensfrage}
	\label{image:Gewissensfrage}
		Download am: 03.03.2020 06:19\\
		URL: https://www.google.de/url?sa=i&url=https\%3A\%2F\%2Fwww.comune.cinisello-balsamo.mi.it\%2Fspip.php\%3Farticle23706&psig=AOvVaw0u\_fu8ynT2FsqbaZykBKeM&ust=1583299071897000&source=images&cd=vfe&ved=0CAIQjRxqFwoTCMCMzOnG\_ecCFQAAAAAdAAAAABAS
% Dokumenteinstellungen und Anpassungen
\input{resources/settings/settings}
\input{resources/styles/adjustments}

\usepackage{graphicx}
\usepackage{subcaption}
\usepackage{float}
\usepackage{color}
\usepackage{listings}
\usepackage{eso-pic}
\usepackage{caption}
\usepackage{csquotes}
\usepackage{pgfplots}

\definecolor{myblue}{HTML}{92dcec}

\newcommand\BackgroundPic{%
\put(0,0){%
\parbox[b][\paperheight]{\paperwidth}{%
\vfill
\centering
\includegraphics[width=\paperwidth,height=\paperheight,%
keepaspectratio]{resources/images/deckblatt.jpg}%
\vfill
}}}


\include{titles/graduation}
% Variablen für das Deckblatt
\gradeType{Abitur 2021}
\germanTitle{Face Recognition mit Hilfe von Künstlicher Intelligenz}
\englishTitle{}
\authorFirstname{Kevin}
\authorLastname{Pagenkämper}
\authorBirthplace{Nordhorn}
\discipline{Technik und Ethik}
\courseOfStudies{Informatik}
\matrikelnr{112}
\submitDate{03.03.2020}
\firstExaminer{Andre Nixdorf}
\secondExaminer{Henrike Schnöing}


\include{titles/eidesstatt}
\place{Nordhorn}


\lstset{literate=%
  {Ö}{{\"O}}1
  {Ä}{{\"A}}1
  {Ü}{{\"U}}1
  {ß}{{\ss}}1
  {ü}{{\"u}}1
  {ä}{{\"a}}1
  {ö}{{\"o}}1
}

% Javascript als Sprache für die lstings
\lstdefinelanguage{JavaScript}{
  keywords={typeof, new, true, false, catch, function, return, null, catch, switch, var, if, in, while, do, else, case, break, that, globals, let, const},
  keywordstyle=\color{blue}\bfseries,
  ndkeywords={class, export, boolean, throw, implements, import, this},
  ndkeywordstyle=\color{darkgray}\bfseries,
  identifierstyle=\color{black},
  sensitive=false,
  comment=[l]{//},
  morecomment=[s]{/*}{*/},
  commentstyle=\color{gray}\ttfamily,
  stringstyle=\color{red}\ttfamily,
  morestring=[b]',
  morestring=[b]"
}
\lstset{
    language=JavaScript,
    numbers=none,
    frame=leftline,
    tabsize=2,
    rulesepcolor=\color{gray},
    rulecolor=\color{black},
    captionpos=b,
    breaklines=true,
    breakatwhitespace=false,
}

% Neuer Befehl um Subsubsubkapitel schreiben zu können
\newcommand{\subsubsubsection}[1]{\paragraph{#1}\mbox{}\\}
\setcounter{secnumdepth}{4}
\setcounter{tocdepth}{4}
\usepackage{fontspec}
\setmainfont{Arial}

\begin{document}

% Entfernen der Seitenzahlen und BackroundPic als Hintergrund nutzen
\pagenumbering{gobble}
\AddToShipoutPicture{\BackgroundPic}

% Titelblatt erzeugen
\maketitle
\newpage

% Eidesstattliche Erklärung erzeugen
\makeeidesstatt
\thispagestyle{empty}

% Abstract und Inhaltsverzeichnisse einfügen (Numerierung römisch ab Inhaltsverzeichnis)
%\input{chapter/0_Abstract.tex}
\ClearShipoutPicture
\thispagestyle{empty}
\newpage
\pagenumbering{Roman}
\tableofcontents
%\newpage
%\listoffigures
\newpage

% Kapitel einfügen
\pagenumbering{arabic}
% „ = Alt + 0132
% “ = Alt + 0147
\section{Einleitung}
\label{sec:einleitung}
Vor ungefähr 64 Jahren wurde das erste Programm, welches speziell zur Problemlösung und Nachahmung eines Menschen entwickelt wurde, von Herbert Alexander Simon und Allen Newell geschrieben. Dieses Programm war Quellen zufolge das erste Programm welches der Definition von Künstlicher Intelligenz, ein Programm welches speziell dafür entwickelt wurde die Fähigkeiten eines Menschen Nachzuahmen, gerecht wurde [vgl. \ref{bib:LogicTheorist}] wobei zu dem Zeitpunkt der Entwicklung dieser das Wort "Künstliche Intelligenz" [folgend auch als KI bezeichnet] noch nicht existierte bzw. nicht in diesem Kontext verwendet wurde [vgl. \ref{LogicTheorist:Wikipedia}]. Seitdem hat sich das Thema KI sowohl in der Hardware aber auch in der Software stark weiterentwickelt. Wo Künstliche Intelligenzen früher nur mit einfachen Zahlen arbeiten konnten, können einige mittlerweile Personen, Gesichter, Objekte und andere Dinge identifizieren und differenzieren. Doch auch hierbei gibt es Probleme die seit der Entwicklung von Gesichtserkennungsprogrammen in den 1960er nicht vollständig gelöst werden konnten, weshalb es bisher auch keine KI gibt welche keine Mängel aufweist. Und gibt es eine KI die momentan eine Erfolgschance von 100\% hat ist dies meist aus dem Grund so, dass sie ihrem Schwachpunkt noch nicht begegnet ist. In dieser Facharbeit wird das Thema der Gesichtserkennung, zu engl. "Face recognition", behandelt. Bei dieser Art der Anwendung von Künstlicher Intelligenz liegt die Schwierigkeit besonders bei Unschärfe, Fragmenten, Verzerrungen und Ähnlichkeiten von Objekten und besonders von Gesichtern. Dies ist z.B. bei Zwillingen der Fall an dem viele Systeme noch heute scheitern. Zudem wird die Theorie von Face Recognition mit Hilfe von Künstlicher Intelligenz erklärt, mit praktischen Beispielen erläutert und auf ethischer Basis analysiert.
\newpage
% Theorie Einführung
\section{Theoretischer Hintergrund}
\label{sec:theorie}
    \subsection{Definition Künstliche Intelligenz}
    \label{subsec:definiton_kuenstliche_intelligenz}
    Der Interesse an dem Thema Künstliche Intelligenz sowie die Einsatzmöglichkeiten steigen stetig an. Künstliche Intelligenz ist ein sehr großes Thema. Für eine grobe Wiedergabe worum es in diesem Thema geht liefert Wikipedia einen guten Einstieg: 
    \textit{\enquote{Künstliche Intelligenz [...] ist ein Teilgebiet der Informatik, welches sich mit der Automatisierung intelligenten Verhaltens und dem maschinellen Lernen befasst.}} [siehe \ref{wiki:KuenstlicheIntelligenz}]\\

    Mit dieser Aussage kann man Künstliche Intelligenz im groben Beschreiben. Etwas genauer beschreibt es Klaus Mainzer in seinem Buch \enquote{Künstliche Intelligenz - Wann übernehme die Maschinen?} [siehe \ref{book:KI_WannUebernehmenDieMaschinen}]. In diesem geht er auf die Definition der Intelligenz von Systemen ein. Laut ihm heißt ein System intelligent \textit{\enquote{wenn es selbstständig und effizient Probleme lösen kann. Der Grad der Intelligenz hängt vom Grad der Selbstständigkeit, dem Grad der Komplexität des Problems und dem Grad der Effizienz des Problemlösungsverfahrens ab}} [siehe S.3 in \ref{book:KI_WannUebernehmenDieMaschinen}].\\
    
    Um es einmal auf den Punkt zu bringen, bezeichnet \enquote{Künstliche Intelligenz} ein System welches selbstständig lernen und effizient Probleme lösen kann und zudem noch lernfähig ist.

    \subsection{Wie funktioniert eine Künstliche Intelligenz?}
    \label{subsec:wie_funktioniert_eine_kuenstliche_intelligenz}
    Künstliche Intelligenzen funktionieren je nach Abwandlung etwas verschieden aber ähneln sich dennoch in ihrem Aufbau. In dieser Arbeit werden die sogenannten Neuronalen Netzwerke, welche in den meisten Systemen Anwendung finden, behandelt. 

    \begin{figure}[ht]
        \includegraphics[width=\textwidth, scale=0.8]{resources/images/img/Neural Networks/neural-network-example.png}
        \caption{Beispiel eines Neuronalen Netzes [Quelle: \ref{subsubsec:Beispiel_eines_Neuronalen_Netzes}]}
        \label{fig:Beispiel_Neuronales_Netz}
    \end{figure}

    Diese sind nach der Theorie wie unser Gehirn funktioniert aufgebaut. Hierbei arbeiten viele Neurone zusammen um Daten und Befehle zu verarbeiten [vgl. \ref{subsubsec:dasgehirn:Zellen-Arbeiter_Des_Gehrins}]. Die Neuronalen Netze besitzen 3 verschiedene Layer welche zur Datenverarbeitung genutzt werden können.\\
    
    Diese Layer werden \enquote{Input Layer}, \enquote{Hidden Layer} und \enquote{Output Layer} genannt. Innerhalb dieser gibt es sogenannte Neuronen, welche mit jeweils jedem Neuron der vorherigen und der nächsten Schicht verbunden ist. Jedes dieser Neuronen hat ein Gewicht welches das Ergebnis im \enquote{Output Layer} beeinflussen kann. Wie viele Neuron-Verbindungen es gibt hängt von der Anzahl der Neuronen sowie der \enquote{Hidden Layers} ab. Im Beispiel [Abbildung \ref{fig:Beispiel_Neuronales_Netz}] ist diese Rechnung noch sehr simpel.
    
    \begin{align}
        3 \cdot 4 = 12 + (4 \cdot 4) = 28 + (4 \cdot 1) = 32
    \end{align}
    
    Das Beispiel hat also insgesamt 32 Neuronen Verbindungen. Dies ist noch ein sehr kleines Netz, welches für Anschauungszwecke allerdings völlig ausreicht.\\
    
        \subsubsection{Trainieren eines Neuronalen Netzes}
        Mit einem gerade erstellten Neuronalen Netz können noch keine Daten sinnvoll verarbeitet werden. Zunächst muss dieses auf das Anwendungsgebiet trainiert werden.\\
        In diesem Training bekommen die Neuronen zunächst ein zufälliges Gewicht welches meist zwischen -1 und 1 liegt. Als nächstes werden Daten in das Neuronale Netz gegeben. Natürlich wird die Ausgabe größtenteils falsch sein und nur durch Glück einen richtigen Ansatz haben. Nach dem der erste Datensatz durchgelaufen ist, passt das Netz seine Gewichte automatisch an. Hierbei erkennt es welche Neuronen den größten Einfluss auf das aktuelle Ergebnis hatten und berechnet dabei die Abweichung vom erwarteten Ergebnis die diese verursachten. Im Anschluss werden diese fehlerhaften Gewichte ein kleines bisschen angepasst sodass die Ergebnisse am \enquote{Output Layer} ein wenig näher am erwarteten Ergebnis sind als vorher [vgl. \ref{subsubsec:Wie_lernen_neuronale_netze}]. Dieser Vorgang wird nun mit einem großen Datensatz mehrere hunderte oder tausend mal wiederholt. Hierbei gilt je größer der Datensatz desto genauer und \enquote{intelligenter [vgl. \ref{subsec:definiton_kuenstliche_intelligenz}]} das Neuronale Netz.
    
    \subsection{Erste Konzepte Künstlicher Intelligenz}
    \label{subsec:Theorie:Erste_Konzepte_von_KI}
        Wie in der Einleitung schon erwähnt, begann die Entwicklung von Künstlicher Intelligenz schon vor mehreren Jahrzehnten. Am Anfang gab es die Bezeichung \enquote{Künstliche Intelligenz} noch nicht. Dennoch schafften es die beiden Wissenschaftler Herbert Alexander Simon und Allen Newell ein Lernfähiges Computersystem zu erschaffen, welches einen menschlich-orientierten Lösungsansatz verwendete [Bezug zu \ref{sec:einleitung}]. Dieses System hieß \enquote{Logic Theorist}
        
        \subsubsection{Logic Theorist - Was konnte es?}
        \label{subsubsec:Logic_Theorist:Was_konnte_es}
            Der Logic Theorist sollte dazu dienen Mathematische Behauptungen zu beweisen. Von dem zweiten Kapitel der \enquote{Prinzipien der Mathematik}, ein Buch Trilogie welche Grundlagen der Mathematik zusammenfasst [siehe \ref{book:prinzipa_of_mathmatics}], konnte das Programm von den ersten 52 Theorien insgesamt 38 beweisen. Einige dieser Lösungen waren laut Quellen sogar \enquote{schöner} gelöst als die handschriftliche Lösung der Autoren Bertrand Russel und Alfred North Whiteheard

    \subsection{Das System der Gesichtserkennung}
\newpage
\section{Praktische Anwendung von Gesichtserkennung}
\label{sec:Praltische_Anwendung_Kuenslicher_Intelligenz}

    Nun sollte alles grundlegende zu der Theoretischen Gesichtserkennung erklärt worden sein. Daher wird in dem folgenden Kapitel auf die Praktische Anwendung von KI und \enquote{face recognition} eingegangen.

    \subsection{Apples \enquote{Face ID}}
    \label{subsec:FaceID_TrustedFace}
        Seit einigen Jahren unterstützen Android sowie Apple Smartphones Trusted-Face bzw. Face ID. Dies sind die offiziellen Namen für die Gesichtserkennungssoftware auf den jeweiligen Geräten.

        \subsubsection{Apples Ansatz mit Face ID}
            Wenn man sein IPhone entsperren möchte und Face ID eingerichtet hat, muss man meist nur einmal nach oben wischen und schon ist das Gerät entsperrt. Dies liegt an zum einen an den Prozessoren, welche von Jahr zu Jahr leistungsstärker werden, und an der Software hinter Face ID, welche so stark optimiert wurde, dass der Nutzer nahezu keine Verzögerung beim Entsperren per Face ID bemerkt.\\
            Face ID funktioniert mit Hilfe einer TrueDepth-Kamera. Diese kann über 30.000 Punkte projizieren und analysieren um daraus eine Tiefenkarte zu erstellen [vgl. \ref{apple:face_id}]. Dies ist vergleichbar zu [\ref{subsubsec:Elastic_bunch_graph_Matching}] wobei hier keine Grauwerte und Jets zur Analyse verwendet werden, sondern rohe Daten. Deshalb trifft die Definition von einem 3D-Verfahren [vgl. \ref{subsubsec:3d_verfahren}] eher auf Face ID zu.\\
            Es gibt mittlerweile aber auch Kritik an der erst so hochgelobten Technik. Es wird davon geredet, dass Face ID zu langsam wäre und Updates die Software verlangsamen [vgl. \ref{maclife:face_id_too_slow}]. Dies ist allerdings nicht bewiesen und einige Nutzer sagen das es langsamer ist als vorher und wieder andere sagen, dass sich nichts geändert hat [vgl. Kommentare in \ref{maclife:face_id_too_slow}]. Man muss bedenken, dass auch wenn Apple versucht Face ID in jeglicher Situation nutzbar zu machen hat die Software dennoch wie jede andere ihre schwächen. Einerseits wäre das Problem, dass man sein Apple Gerät mit dieser Methode nur im Portrait Modus entsperren kann. Außerdem soll es Probleme mit Ski-Brillen geben da diese zu viel vom Gesicht überdecken.

    \subsection{Die Massenüberwachung in China}
    \label{China:Massenueberwachung}
        China hat sich in den letzten Jahren immer mehr auf KI spezialisiert. Doch China möchte nicht nur eine der größten Länder im Bereich face recognition werden, sondern möchte seine Bevölkerung auch unter eine komplette Kontrolle bringen [vgl. \ref{berlinerZeitung:totale_ueberwachung_in_china}].

        \subsubsection{Das Sozialpunktesystem}
        \label{china:das_sozialpunktesystem}
            Mittlerweile hat China ein System eingerichtet um jeden Bürger individuell zu bewerten. Man kann das Land mit einem Club in einem Spiel vergleichen. Verhält man sich ruhig bzw. hält sich an die Regeln hat man keine Konsequenzen zu fürchten. Engagiert man sich dazu auch noch für den Club, so kann es sein das man dafür belohnt wird. Verhält man sich hingegen negativ gegenüber seines Clubs kann man absteigen und verliert seine Rechte.\\
            Es klingt zunächst extrem dargestellt, aber wie man in [\ref{welt:China_ki_weltmacht}] lesen kann wurden Jahr 2018 ca. 17.5 Millionen Flugtickets und 5.5 Millionen Bahntickets abgelehnt bzw. nicht ausgestellt worden sein. \enquote{Begründung: zu wenige Sozialkreditpunkte} [Zitat aus \ref{welt:China_ki_weltmacht}].

        \subsubsection{Vor- und Nachteile dieser Art der Überwachung}
            Auch wenn das Sozialpunktesystem aus [\ref{china:das_sozialpunktesystem}] weitgehend umstritten ist befürworten es die meisten der Chinesen. Viele begründen dies damit, dass sich ihr Verhalten gegenüber anderen verbessert zu haben scheint. Außerdem sei es einigen sogar ganz egal, dass dieses System genutzt wird: \enquote{Die Leute sind ohnehin daran gewöhnt, dass alles kontrolliert wird, sagt Kostka. Da ist der Sprung, dass die Regierung auf diese Weise Daten sammelt, nicht so groß.} [Zitat aus \ref{handelsblatt:chinesen_schaetzen_die_ueberwachung}].

    \subsection{Gesichtserkennung mit Open-Source-Projekten}
        Um eine Gesichtserkennung auf ein Bild oder einen Video stream anzuwenden benötigt es mittlerweile kein großes Wissen wie eine KI programmiert wird. Es gibt in vielerlei Sprachen, z.B. Python, C#, C++ oder Java, für die verschiedensten Systeme, z.B. Linux, Windows, Mac OS, Raspberry Pi oder anderen Development-Board. Ein schönes Beispiel ist hierbei die Quelle [\ref{realpython:face_recognition_in_under_25_lines_of_code}] in der mit Hilfe der relativ bekannten Libary OpenCV und der Programmiersprache Python eine Gesichtserkennungssoftware in unter 25 Zeilen Code geschrieben wird. Dabei wird Schritt für Schritt erklärt, was zu tun ist und wie die Funktionen funktionieren.

\newpage
\section{Hinterfragen der Ethischen korrektheit}
\newpage
\section{Literatur Verzeichnis}
\label{sec:literatur}

\subsection{Logic Theorist}
\label{bib:LogicTheorist}
	Letzter Zugriff: 12.02.2020\\
	URL: https://history-computer.com/ModernComputer/Software/LogicTheorist.html

	\subsubsection{Memorandum}
	\label{LogicTheorist:Memorandum}
		Letzter Zugriff: 18.02.2020 12:38\\
		URL: https://history-computer.com/Library/Logic\%20Theorist\%20memorandum.pdf
		
	\subsubsection{Wikipedia Artikel}
	\label{LogicTheorist:Wikipedia}
	Letzter Zugriff: 18.02.2020\\
	URL: https://en.wikipedia.org/wiki/Logic\_Theorist

\subsection{Allgemeine Quellen zum Thema KI}
\label{bib:AllgeimeineQuellen}
		\subsubsection{Wikipedia Artikel zum Thema Künstliche Intelligenz}
		\label{wiki:KuenstlicheIntelligenz}
		Letzter Zugriff: 19.02.2020 12:25\\
		URL: https://de.wikipedia.org/wiki/Künstliche\_Intelligenz

		\subsubsection{Gesichtserkennung}
		\label{wiki:face_recognition}
		Letzter Zugriff: 02.03.2020 17:19\\
		URL: https://de.wikipedia.org/wiki/Gesichtserkennung
	
\subsection{Aufbau und Funktionsweise Neuronaler Netze}		
\label{subsec:Aufbau_Funktion_Neuronaler_Netze}
		\subsubsection{JAAI - Aufbau von Neuronalen Netzen}
		\label{subsubsec:Aufbau_von_Neuronalen_Netzen}
		Letzter Zugriff: 22.02.22 15:04\\
		URL: https://jaai.de/kuenstliche-neuronale-netze-aufbau-funktion-291/

		\subsubsection{Wie lernen Neuronale Netze?}
		\label{subsubsec:Wie_lernen_neuronale_netze}
		Letzter Zugriff: 23.02.2020 17:49\\
		URL: https://jaai.de/machine-deep-learning-529/

\subsection{Bücher}
\label{books}

	\subsubsection{Buch: Digitale Gesichtserkennung}
	\label{book:DigitalGesichtserkennung}
		Untertitel: Theoretischer Überblick und praktische C++-Implementierung\\
		Autor: Andreas G. Ranftl\\
		Jahr: 2012\\
		Ort: Hamburg\\
		ISBN: 978-3-86341-432-0\\

	\subsubsection{Buch: Künstliche Intelligenz - Wann übernehmen die Maschinen?}
	\label{book:KI_WannUebernehmenDieMaschinen}
		Autor: Klaus Mainzer\\
		Jahr: 2016\\
		ISBN: 978-3-662-48452-4\\
		ISBN: 978-3-662-48453-1 (eBook)

	\subsubsection{Buch: Prinzipien der Mathematik}
	\label{book:prinzipa_of_mathmatics}
		Letzter Zugriff: 24.02.2020 12:14\\
		Autor: Bertrand Russel, Alfred North Whitehead\\
		URL: https://plato.stanford.edu/entries/principia-mathematica/

\subsection{Informationen über das menschliche Gehirn}
\label{subsec:Informationen_ueber_das_menschliche_Gehirn}
	\subsubsection{dasgehirn.info - Zellen: spezialisierte Arbeiter des Gehirns}
	\label{subsubsec:dasgehirn:Zellen-Arbeiter_Des_Gehrins}
	Letzter Zugriff: 22.02.2020 18:03\\
	URL: https://www.dasgehirn.info/grundlagen/kommunikation-der-zellen/zellen-spezialisierte-arbeiter-des-gehirns

\subsection{Viola Jones Algorithmus}
\label{algoryhtm:viola_jones}
	Letzter Zugriff: 02.03.2020 12:52\\
	URL: https://pdfs.semanticscholar.org/40b1/0e330a5511a6a45f42c8b86da222504c717f.pdf

\subsection{Facial Recognition}
\label{source:searchenterpriseai:facial_recognition}
	Letzter Zugriff: 02.03.2020 13:47\\
	URL: https://searchenterpriseai.techtarget.com/definition/facial-recognition

	\subsubsection{Elastic Bunch Graph Matching}
	\label{scholarpedia:ebgm}
	Letzter Zugriff: 02.03.2020 19:42\\
	URL: http://www.scholarpedia.org/article/Elastic\_Bunch\_Graph_Matching

	\subsubsection{Gabor_wavelets}
	\label{Gabor_wavelets}
	Letzter Zugriff: 02.03.2020 20:53\\
	URL: https://en.wikipedia.org/wiki/Gabor\_wavelet

\subsection{\enquote{Machine Learning} und \enquote{Deep Learning}}
\label{subsec:machine_learning_and_deep_learning}
	Letzter Zugriff: 01.03.2020 22:22\\
	URL: https://www.zendesk.com/blog/machine-learning-and-deep-learning/

\subsection{Netflix Research}
\label{Netlfix_Research}
	\subsubsection{Machine Learning - Learn hoe to entertain the World}
	\label{Netflix_Research:machine_learning_learn_how_to_entertain_the_world}
	Letzter Zugriff: 02.03.2020\\
	URL: https://research.netflix.com/research-area/machine-learning

\subsection{Dynamic Link Machine}
\label{subsec:dynamic_link_machine}
	Letzter Zugriff: 02.03.2020 22:55\\
	URL: http://www.cs.utexas.edu/users/nn/web-pubs/htmlbook96/

\subsection{Apples Support - Fortschritt von Face ID}
\label{apple:face_id}
	Letzter Zugriff: 03.03.2020 03:28\\
	URL: https://support.apple.com/de-de/HT208108

\subsection{Face ID - Face ID zu langsam?}
\label{maclife:face_id_too_slow}
	Letzter Zugriff: 03.03.2020 04:00\\
	URL: https://www.maclife.de/news/entsperrt-face-id-iphone-schnell-genug-oder-langsam-100111801.html

	\subsubsection{Vergleich von Touch ID und Face ID}
	\label{cickrepair:touch_id_vs_face_id}
		Letzter Zugriff: 03.03.2020 04:08\\
		URL: https://www.clickrepair.de/ratgeber/ratgeber-iphone/ratgeber-iphone-tipps/vergleich-von-face-id-und-touch-id-was-ist-besser

\subsection{Massenüberwachungsstaat China}
	\subsubsection{Welt.de: Ein Sputnik-Moment macht China zur KI-Weltmacht}
	\label{welt:China_ki_weltmacht}
		Letzter Zugriff: 03.03.2020 04:19\\
		URL: https://www.welt.de/kultur/article191734655/Wie-China-mit-kuenstlicher-Intelligenz-zum-Ueberwachungsstaat-wird.html

	\subsubsection{Berliner Zeitung - Totale Überwachung in China}
	\label{berlinerZeitung:totale_ueberwachung_in_china}
		Letzter Zugriff: 03.03.2020 04:21\\
		URL: https://www.berliner-zeitung.de/zukunft-technologie/kuenstliche-intelligenz-totale-ueberwachung-ist-in-china-laengst-normalitaet-li.37733

\subsection{Bilder}
\label{subsec:Bilder_Anhang}
	\subsubsection{Beispiel eines Neuronalen Netzes}
	\label{subsubsec:Beispiel_eines_Neuronalen_Netzes}
	Download am: 22.02.2020 15:20\\
	URL: https://de.cleanpng.com/png-jxw2np/

	\subsubsection{Datenpunkte eines EGBM-Algorithmus}
	\label{image:datapoints_of_a_EGBM_algoryhtm}
	Screenshot am: 02.03.2020 19:17\\
	Von URL: http://www.scholarpedia.org/article/Elastic\_Bunch\_Graph\_Matching

	\subsubsection{Bunch of Jets}
	\label{image:Bunch_of_jets}
		Download am: 03.03.2020 00:56\\
		URL: http://www.scholarpedia.org/article/File\:BunchGraph.png


% Bibliografie im .bib Format einfügen
\bibliography{bachelorarbeit}

\end{document}
